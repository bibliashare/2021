%-----------------------------------------------------------------------------------------------
\documentclass[12pt,aspectratio=169]{beamer}
%-----------------------------------------------------------------------------------------------
\usepackage{pslatex}
\usepackage[greek,french,english,brazil]{babel} % last becomes the active one
\usepackage{ulem}
\renewcommand{\ULdepth}{3.0pt}
%-----------------------------------------------------------------------------------------------
\newcommand{\YA}{%
    \mbox{%
        Y\makebox[0pt][l]{\hspace{-0.178em}\raisebox{-0.00ex}{\scalebox{0.30}{E}}}%
        H\makebox[0pt][l]{\hspace{-0.010em}\raisebox{-0.00ex}{\scalebox{0.30}{O}}}%
        W\makebox[0pt][l]{\hspace{-0.245em}\raisebox{-0.00ex}{\scalebox{0.30}{A}}}%
        H%
    }%
}
%-----------------------------------------------------------------------------------------------
\newcommand{\ver}[1]{%
    \raisebox{0.50ex}{%
        \scalebox{1.1}{%
            \pmb{\textbf{\textcolor{BSpbg}{#1}}}%
        }%
    }%
}
%-----------------------------------------------------------------------------------------------
\newcommand{\QUOTE}[1]{%
    \par\noindent\hspace*{0.1\linewidth}%
    \begin{minipage}{0.8\linewidth}%
        \linespread{1.35}\large{#1}%
    \end{minipage}%
}
%-----------------------------------------------------------------------------------------------
\newcommand{\RED}[1]{{\textcolor{TXred}{#1}}}
\newcommand{\ORA}[1]{{\textcolor{TXora}{#1}}}
\newcommand{\YEL}[1]{{\textcolor{TXyel}{#1}}}
\newcommand{\GRE}[1]{{\textcolor{TXgre}{#1}}}
\newcommand{\CYA}[1]{{\textcolor{TXcya}{#1}}}
\newcommand{\BLU}[1]{{\textcolor{TXblu}{#1}}}
\newcommand{\MAG}[1]{{\textcolor{TXmag}{#1}}}
\newcommand{\BRI}[1]{{\textcolor{BSpbg}{#1}}}   % Bright
%-----------------------------------------------------------------------------------------------
\newcommand{\ENtxt}[1]{\begin{otherlanguage}{english}{{#1}}\end{otherlanguage}}
\newcommand{\GRtxt}[1]{\begin{otherlanguage}{greek}{{#1}}\end{otherlanguage}}
\newcommand{\FRtxt}[1]{\begin{otherlanguage}{french}{{#1}}\end{otherlanguage}}
%-----------------------------------------------------------------------------------------------
\usetheme{CambridgeUS}
\usefonttheme{serif}
\usecolortheme{BShare1}
%-----------------------------------------------------------------------------------------------
\title{O Sal da Terra}
\subtitle{Uma Tradução de Mt 5.13}
\author{Bíblia Share}
%\institute{Bíblia Share}
\date[{\tiny\tt\today}]{{\scriptsize\tt%
    \includegraphics[height=6.0mm]{res/cc/by-nc-nd-88x31.pdf}\\[\smallskipamount]
    \today
}}
%-----------------------------------------------------------------------------------------------
\begin{document}
%-----------------------------------------------------------------------------------------------
\logo{%
    \parbox{158mm}{% There's a 1mm gap on each side of the 160mm x 90mm slide logo line
    \mode<beamer>{%
        \hfill\includegraphics[height=9.0mm]{res/logo/BibliaShare.pdf}%
    }
    \mode<handout>{%
        \hfill\includegraphics[height=9.0mm]{res/logo/BibliaShare.pdf}%
    }
}}
%-----------------------------------------------------------------------------------------------
\begin{frame}
    \titlepage
\end{frame}
%-----------------------------------------------------------------------------------------------
\section{Verso Base}
%-----------------------------------------------------------------------------------------------

    \begin{frame}{\BRI{Mt 5.13 (ARA)}}
        \QUOTE{%
            %-----!j 92 -i12
            \ver{13}~%
            \alt<1-1>{Vós}{\YEL{Vós}} sois o
            \alt<1-2>{sal}{\GRE{sal}} da terra; ora, se o
            \alt<1-3>{sal}{\GRE{sal}} vier a ser
            \alt<1-4>{insípido}{\GRE{insípido}}, como lhe
            \alt<1-5>{restaurar}{\YEL{restaurar}} o
            \alt<1-6>{sabor}{\GRE{sabor}}? Para nada mais
            \alt<1-7>{presta}{\YEL{presta}} senão para,
            \alt<1-8>{lançado fora}{\CYA{lançado fora}}, ser
            \alt<1-9>{pisado}{}\only<10>{\CYA{pisado}} pelos homens.
        }
    \end{frame}

%-----------------------------------------------------------------------------------------------
\section{Termos-Chave}
%-----------------------------------------------------------------------------------------------

    \begin{frame}{\BRI{Mt 5.13 (ARA)}}
        \QUOTE{%
            %-----!j 92 -i12
            \ver{13}~%
            \YEL{Vós} sois o
            \GRE{sal} da terra; ora, se o
            \GRE{sal}
            \alt<1-1>%
                {vier a ser \GRE{insípido}}%
                {\uuline{vier a ser \GRE{insípido}}},
            \alt<1-2>%
                {como lhe \YEL{restaurar} o \GRE{sabor}}{}%
                \only<3>{\uuline{como lhe \YEL{restaurar} o \GRE{sabor}}}?
            Para nada mais
            \YEL{presta} senão para,
            \CYA{lançado fora}, ser
            \CYA{pisado} pelos homens.
        }
    \end{frame}

    \subsection{Vier a ser \textit{insípido\/}}

    \begin{frame}{\BRI{``Vier a ser insípido''} -- Outras Versões}
        \begin{itemize}
            \item<1-> \YEL{Perder} / \YEL{perda}:
                \begin{itemize}
                    \item \YEL{do sabor} / \YEL{do gosto}:
                        \BRI{\textbf{AECR, AM, AP, BJC, BKJ, CNBB, EP, IAGO, ISBB, MD, NBV, NVI,
                            NVT, NTLH, PER, POV, SBU, TEB, VFL, VOZ}};
                    \item \YEL{de qualidades, da capacidade}:
                        \BRI{\textbf{A21, MESS}};
                    \item \YEL{da força}:
                        \BRI{\textbf{APF, TNMES}}.
                \end{itemize}
            \item<2-> \YEL{Se tornar} / \YEL{for} / \YEL{vier a ser}:
                \begin{itemize}
                    \item \YEL{insípido}:
                        \BRI{\textbf{ACC, ACF, AEC, ARA, ARC, FL}} (com \MAG{literal} no rodapé)%
                        \BRI{\textbf{, IWCL, NAA, TB}};
                    \item \YEL{insosso}:
                        \BRI{\textbf{BJ, HDD}} (com \MAG{literal} no rodapé).
                \end{itemize}
        \end{itemize}
    \end{frame}

    \begin{frame}{\BRI{``Vier a ser insípido''} -- \GRtxt{mwranj\~h|}}
        \begin{itemize}
            \item<1-> \BRI{\textbf{FL}} (rodapé): \MAG{``se o sal se tornar imbecil''};
                \\[\medskipamount]
            \item<2-> \BRI{\textbf{HDD}} (rodapé): \MAG{``se, porém, o sal enlouquecer''} ou
                \MAG{``tornar-se tolo''};
                \\[\medskipamount]
            \item<3-> ... cuja estrutura segue o original: \YEL{\GRtxt{>ean, d`e, t`o <'alas}},
                lit.:~\MAG{``se, porém, o sal''} ...
                \\[\medskipamount]
            \item<4-> seguido do termo-chave \BRI{\GRtxt{mwranj\~h|}}, o \GRE{aoristo},
                \GRE{passivo}, \GRE{subjuntivo} de
                \BRI{\GRtxt{mwr'ainw}}~\cite{2013-MounceWD-VidaNova,
                2007-FribergB+FribergT-VidaNova, 2009-MounceWD-Vida}; % p.~350
                \\[\medskipamount]
            \item<5-> o qual, no NT, trans.,~significa: \CYA{``tornar louco''},
                \CYA{``demonstrar a tolice''},
                \\[\medskipamount]
            \item<6-> o que já explica as variantes entre \BRI{\textbf{FL}} e \BRI{\textbf{HDD}}
                acima.
                \\[\medskipamount]
            \item<7-> O \GRE{aor.~subj.}~modifica o verbo em: \CYA{``}(se) \CYA{vier a''}, visto
                que contempla uma \BRI{\textit{possibilidade}} (\GRE{subj.}),~\BRI{\textit{sem
                aspecto}} definido, ou sua \BRI{\textit{ocorrência}} ou
                \BRI{\textit{incipiência}} (\GRE{aor.}).
                \\[\medskipamount]
            \item<8-> No português, \MAG{``se o sal vier a enlouquecer''} tem valor
                \GRE{passivo}, diferentemente de \ \ \RED{``se o sal ficar enlouquecedor''}.
        \end{itemize}
    \end{frame}

    \begin{frame}{\BRI{``Vier a ser insípido''} -- \GRtxt{mwranj\~h|}}
        \begin{itemize}
            \item 1
        \end{itemize}
    \end{frame}

    \subsection{Como lhe \textit{restaurar o sabor\/}?}

    \begin{frame}{\BRI{``Como lhe restaurar o sabor?''} -- Outras Versões}
        \begin{itemize}
            \item<1-> \RED{Deixa de ser sal}:
                \BRI{\textbf{NTLH}}
                $\therefore$ desvio!!
            \item<1-> \RED{Como [...] pessoas [...] sentir o tempero}:
                \BRI{\textbf{MESS}}
                $\therefore$ desvio!!
            \item<2-> \ORA{Com o que} / \ORA{outra coisa} (esp.~\BRI{\bf APF})
                $\therefore$ recurso substitutivo ao sal
                \begin{itemize}
                    \item \ORA{salgar} / \ORA{temperar}:
                        \BRI{\textbf{ACC, ACF, AEC, AECR, AP, ARC, BKJ, CNBB, FL, HDD, POV, SBU,
                            VOZ}};
                \end{itemize}
            \item<3-> \YEL{Como} / \YEL{é possível} (\BRI{\bf NVT}):
                $\therefore$ método para o sal
                \begin{itemize}
                    \item \YEL{tornar a ser}:
                        \BRI{\textbf{TEB}}
                        $\therefore$ ser;
                    \item \YEL{restabelecer}:
                        \BRI{\textbf{A21, ARA, NAA, NBV, NVI, NVT, TB, TNMES, VFL}}
                        $\therefore$ atributo.
                \end{itemize}
            \item<3-> \YEL{Com que}
                $\therefore$ recurso para o sal
                \begin{itemize}
                    \item \YEL{salgá-lo}:
                        \BRI{\textbf{BJ, BJC, EP, IAGO, ISBB, PER}};
                    \item \YEL{recuperar, restituir}:
                        \BRI{\textbf{AM, MD}};
                \end{itemize}
        \end{itemize}
    \end{frame}

    \begin{frame}{\BRI{``Como lhe restaurar o sabor?''} -- \GRtxt{>en t'ini <alisj'hsetai?}}
        \begin{itemize}
            \item<1-> \MAG{Em que será salgado?}:
                \BRI{\textbf{IWCL}}
                $\therefore$ recurso imersivo (divino) para o sal
                \\[\medskipamount]
            \item<2-> A tradução \GRE{mais literal} é a que \GRE{melhor se harmoniza} com o NT:
                \\[\medskipamount]
            \item<3-> O Senhor já sabia e o NT revela que alguém é salgado \MAG{em Jesus
                Cristo}:
                \begin{itemize}
                    \item<4-> ``Agora, pois, já nenhuma condenação há para os que \MAG{estão em
                        Cristo Jesus}.'' -- \BRI{Rm~8.1};
                        \\[\smallskipamount]
                    \item<5-> ``E, assim, se alguém \MAG{está em Cristo}, é nova criatura;'' --
                        \BRI{2Co~5.17};
                        \\[\smallskipamount]
                    \item<6-> ``sendo \YEL{justificados} gratuitamente, por sua graça, mediante
                        a \YEL{redenção} que \MAG{há em Cristo Jesus},'' -- \BRI{Rm~3.24};
                        \\[\smallskipamount]
                    \item<7-> ``Porque, assim como, \RED{em Adão, todos morrem}, assim também
                        todos serão \MAG{vivificados em Cristo}.'' -- \BRI{1Co~15.22}.
                \end{itemize}
        \end{itemize}
    \end{frame}

%-----------------------------------------------------------------------------------------------
\section{Tradução Própria}
%-----------------------------------------------------------------------------------------------

%-----------------------------------------------------------------------------------------------
\section{Paralelos e Aplicação}
%-----------------------------------------------------------------------------------------------

%-----------------------------------------------------------------------------------------------
\section{Referências}
%-----------------------------------------------------------------------------------------------

    %------------------------------------------------------------------------------------------
    \begin{frame}[allowframebreaks]{Referências -- }
        \bibliographystyle{unsrt}
        \setbeamertemplate{bibliography item}{\insertbiblabel}
        \bibliography{bibfile.bib}
    \end{frame}
    %------------------------------------------------------------------------------------------

%-----------------------------------------------------------------------------------------------
\end{document}
%-----------------------------------------------------------------------------------------------
